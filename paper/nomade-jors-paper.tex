\documentclass[11pt,a4paper]{article}

\usepackage[utf8]{inputenc}
\usepackage[T1]{fontenc}
\usepackage{hyperref}
\usepackage{graphicx}
\usepackage[margin=2.5cm]{geometry}
\usepackage{parskip}

\title{Software paper for submission to the\\Journal of Open Research Software}
\author{}
\date{}

\begin{document}

\maketitle

\section*{(1) Overview}

\subsection*{Title}
NØMADE: Lightweight HPC Monitoring with Machine Learning-Based Failure Prediction

\subsection*{Paper Authors}
\begin{enumerate}
\item Tonini, João Filipe Riva (corresponding author)
\end{enumerate}

\subsection*{Paper Author Roles and Affiliations}
\begin{enumerate}
\item Academic Research Computing, University of Richmond, Richmond, VA 23173, USA.\\
Email: jtonini@richmond.edu\\
ORCID: 0000-0002-4730-3805
\end{enumerate}

\subsection*{Abstract}
NØMADE (NOde MAnagement DEvice) is a lightweight monitoring and predictive analytics tool for High-Performance Computing (HPC) clusters. It collects system metrics from SLURM-managed environments, stores time-series data in SQLite, and employs machine learning to predict job failures before they occur. A key innovation is applying biogeographical network analysis concepts to HPC monitoring, treating resource domains as interconnected regions where failures cluster at domain boundaries. The tool provides a real-time web dashboard and supports alerts via email, Slack, or webhooks. NØMADE requires no external databases or complex infrastructure, making it suitable for small-to-medium HPC centers and research groups.

\subsection*{Keywords}
HPC; high-performance computing; monitoring; machine learning; SLURM; predictive analytics; failure prediction

\subsection*{Introduction}
HPC administrators face a persistent challenge: detecting job failures before they impact researchers. Enterprise monitoring solutions like Prometheus, Grafana, or Nagios require significant infrastructure and target general IT systems rather than HPC-specific workloads. Existing HPC tools such as TACC Stats [1], XDMoD [2], and LLNL's Lightweight Distributed Metric Service [3] provide detailed metrics but require substantial deployment effort and focus on post-hoc analysis rather than real-time prediction.

Common HPC failure patterns include NFS saturation (jobs writing to network storage instead of local scratch), memory leaks (gradual consumption leading to out-of-memory kills), GPU thermal throttling (temperature-induced performance degradation), and queue starvation (resource contention causing excessive wait times). These failures often exhibit warning signs minutes to hours before critical thresholds are breached.

NØMADE addresses this gap by providing: (1) zero-infrastructure deployment using a single SQLite database with no external services; (2) real-time prediction where an ML ensemble identifies high-risk jobs before failure; (3) predictive alerts through derivative analysis that detects accelerating resource consumption; and (4) domain-aware analysis that recognizes HPC-specific failure patterns at resource boundaries.

A key innovation is the application of biogeographical network analysis concepts to HPC monitoring. Inspired by methods for identifying transition zones between bioregions [4], NØMADE treats HPC resource domains (compute, storage, network) as interconnected regions where failures cluster at domain boundaries---such as transitions between local scratch and network-attached storage (NAS), or between CPU and GPU workloads. This enables failure pattern recognition that emerges from the data rather than from predefined rules.

\subsection*{Implementation and Architecture}
NØMADE is implemented in Python and follows a modular architecture with four main components:

\textbf{Collectors} gather metrics from system tools (\texttt{iostat}, \texttt{vmstat}, \texttt{nvidia-smi}), SLURM commands (\texttt{sacct}, \texttt{squeue}, \texttt{sinfo}), and per-job I/O statistics from \texttt{/proc/[pid]/io}. A SLURM prolog hook captures job context at submission time.

\textbf{Feature Engineering} transforms raw metrics into a 17-dimensional feature vector per job, including CPU and memory efficiency from \texttt{sacct}, NFS write ratios from the job monitor, and system-level indicators (I/O wait, memory pressure, swap activity). These features enable similarity-based analysis across jobs using cosine similarity.

\textbf{ML Prediction} uses an ensemble of three models: a Graph Neural Network (GNN) that captures relationships between similar jobs based on cosine similarity of feature vectors; an LSTM that detects temporal patterns and early warning trajectories; and an Autoencoder that identifies anomalous jobs deviating from normal behavior. The ensemble outputs a continuous risk score (0--1) rather than binary classification.

\textbf{Alert System} supports both threshold-based alerts (disk usage, GPU temperature) and predictive alerts using derivative analysis. When the rate of change indicates a threshold will be breached, alerts fire before the actual breach occurs. Notifications route through email, Slack, or webhooks with configurable cooldowns.

\subsection*{Quality Control}
NØMADE includes a comprehensive test suite covering unit tests for collectors, feature engineering, and ML components; integration tests for the full data pipeline; a demo mode (\texttt{nomade demo}) that generates synthetic HPC data for testing without requiring a real cluster; and continuous integration via GitHub Actions.

The software has been tested on clusters ranging from 6 to 200 nodes, including both CPU-only and GPU-enabled partitions. The codebase follows PEP 8 style guidelines and includes type hints for improved maintainability.

\section*{(2) Availability}

\subsection*{Operating System}
Linux (tested on Ubuntu 22.04, Rocky Linux 9, CentOS 7)

\subsection*{Programming Language}
Python 3.9+

\subsection*{Additional System Requirements}
\begin{itemize}
\item SLURM workload manager (optional, for HPC integration)
\item Network access for dashboard (default port 5000)
\item Minimal disk space for SQLite database ($\sim$1MB per 10,000 jobs)
\end{itemize}

\subsection*{Dependencies}
\begin{itemize}
\item click $\geq$ 8.0
\item toml $\geq$ 0.10
\item numpy $\geq$ 1.21
\item pandas $\geq$ 1.3
\item scipy $\geq$ 1.7
\item scikit-learn (optional, for ML features)
\item torch (optional, for GNN/LSTM models)
\item torch-geometric (optional, for graph neural networks)
\end{itemize}

\subsection*{List of Contributors}
\begin{enumerate}
\item Tonini, João Filipe Riva (Lead developer)
\end{enumerate}

\subsection*{Software Location}

\subsubsection*{Archive}
\begin{tabular}{ll}
Name: & PyPI (Python Package Index) \\
Persistent identifier: & \url{https://pypi.org/project/nomade-hpc/0.3.3/} \\
Licence: & AGPL-3.0-or-later \\
Publisher: & João Filipe Riva Tonini \\
Date published: & December 2025 \\
\end{tabular}

\subsubsection*{Code Repository}
\begin{tabular}{ll}
Name: & GitHub \\
Identifier: & \url{https://github.com/jtonini/nomade} \\
Licence: & AGPL-3.0-or-later \\
Date published: & December 2025 \\
\end{tabular}

\subsection*{Language}
English

\section*{(3) Reuse Potential}

NØMADE is designed for broad reuse across HPC environments of varying scales. The software can be deployed in several contexts:

\textbf{Small Research Groups}: Groups managing individual workstations or small clusters can use NØMADE's zero-infrastructure design to implement sophisticated monitoring without enterprise tools. The \texttt{nomade demo} mode allows evaluation without requiring existing HPC infrastructure.

\textbf{Medium HPC Centers}: University and institutional HPC centers can deploy NØMADE as either a primary monitoring solution or as a complement to existing tools like Grafana or Nagios, adding predictive capabilities to traditional threshold-based alerting.

\textbf{Research Applications}: The biogeographical approach to job similarity analysis provides a novel framework for studying HPC failure patterns. The community data export feature (\texttt{nomade community export}) enables cross-institutional research by generating anonymized datasets suitable for collaborative analysis.

\textbf{Educational Use}: The demo mode and clear architecture make NØMADE suitable for teaching HPC administration and monitoring concepts.

\subsection*{Extension Points}
Developers can extend NØMADE in several ways:
\begin{itemize}
\item \textbf{Custom Collectors}: Add new metric sources by implementing the collector interface
\item \textbf{Alert Channels}: Integrate additional notification services beyond email/Slack/webhooks
\item \textbf{ML Models}: Train site-specific models using the provided feature engineering pipeline
\item \textbf{Dashboard Widgets}: Extend the Flask-based dashboard with custom visualizations
\end{itemize}

\subsection*{Support}
\begin{itemize}
\item GitHub Issues: \url{https://github.com/jtonini/nomade/issues}
\item Email: jtonini@richmond.edu
\item Documentation: \url{https://github.com/jtonini/nomade\#readme}
\end{itemize}

Users and developers are welcome to submit bug reports, feature requests, and pull requests via GitHub.

\section*{Acknowledgements}
The author thanks George Flanagin for advice and inspiration on HPC system administration, and the University of Richmond's Office of the Provost for financial and resource support.

\section*{Funding Statement}
This work was supported by the University of Richmond.

\section*{Competing Interests}
The author declares no competing interests.

\begin{thebibliography}{9}

\bibitem{tacc}
Evans, R.T., Browne, J.C., Barth, W.L. (2014). Comprehensive resource use monitoring for HPC systems with TACC Stats. In: Proceedings of the First International Workshop on HPC User Support Tools, IEEE, pp. 13--21. DOI: \url{https://doi.org/10.1109/SC.2014.18}

\bibitem{xdmod}
Palmer, J.T., Gallo, S.M., Furlani, T.R., Jones, M.D., DeLeon, R.L., White, J.P., Simakov, N., Patra, A.K., Sperhac, J., Yearke, T., Rathsam, R., Inber, M., Guillen, O., Cornelius, C.D. (2015). Open XDMoD: A tool for the comprehensive management of high-performance computing resources. Computing in Science \& Engineering 17(4):52--62. DOI: \url{https://doi.org/10.1109/MCSE.2015.32}

\bibitem{ldms}
Agelastos, A., Allan, B., Brandt, J., Cassella, P., Enos, J., Fullop, J., Gentile, A., Monk, S., Naksinehaboon, N., Ogden, J., Rajan, M., Showerman, M., Stevenson, J., Taerat, N., Tucker, T. (2014). The Lightweight Distributed Metric Service: A scalable infrastructure for continuous monitoring of large scale computing systems and applications. In: Proceedings of the International Conference for High Performance Computing, Networking, Storage and Analysis, IEEE, pp. 154--165. DOI: \url{https://doi.org/10.1145/2616498.2616533}

\bibitem{vilhena}
Vilhena, D.A., Antonelli, A. (2015). A network approach for identifying and delimiting biogeographical regions. Nature Communications 6:6848. DOI: \url{https://doi.org/10.1038/ncomms7848}

\bibitem{sklearn}
Pedregosa, F., Varoquaux, G., Gramfort, A., Michel, V., Thirion, B., Grisel, O., Blondel, M., Prettenhofer, P., Weiss, R., Dubourg, V., Vanderplas, J., Passos, A., Cournapeau, D., Brucher, M., Perrot, M., Duchesnay, E. (2011). Scikit-learn: Machine learning in Python. Journal of Machine Learning Research 12:2825--2830. URL: \url{http://jmlr.org/papers/v12/pedregosa11a.html}

\bibitem{slurm}
Yoo, A.B., Jette, M.A., Grondona, M. (2003). SLURM: Simple Linux Utility for Resource Management. In: Job Scheduling Strategies for Parallel Processing, Lecture Notes in Computer Science, vol. 2862, Springer, pp. 44--60. DOI: \url{https://doi.org/10.1007/10968987_3}

\end{thebibliography}

\end{document}
